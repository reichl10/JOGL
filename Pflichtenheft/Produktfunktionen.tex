

\chapter{Produktfunktionen}
\section{Fensterverhalten}
\begin{itemize}
\item Das Programmfenster startet mit einer Größe von 1024x768 Pixeln. Die Größe ist mit 800x600 Pixeln nach unten, aber nicht nach oben beschränkt und kann vom Benutzer beliebig  verändert werden.
\item Die Anwendung bietet einen Vollbildmodus, in dem Kartenansicht mit Seitenleiste über den gesamten Bildschirm gestreckt werden und Fensterdekorationen entfallen. Zwischen Fenster- und Vollbildmodus kann mit dem Tastenkürzel F11 gewechselt werden.
\item Mit dem Tastenkürzel Strg+Q wird die Anwendung ohne Rückfrage beendet.
\end{itemize}
\section{Kartenansicht}
\begin{itemize}
\item Die Kartenansicht kann interaktiv mit Maus oder Tastatur bedient werden.
\item Im Sonnensystemmodus 
\item Im 2D-Kartenmodus kann die Ansicht nach links/rechts und oben/unten verschoben sowie (perspektivisch) gekippt werden. 
\item Im 3D-Kartenmodus kann die Ansicht kann um die Erdachse sowie in Richtung der Pole gedreht; ab einem gewissen Zoomlevel am Kameraursprung gekippt werden.
\item Mit Linksklick-Ziehen oder den Pfeiltasten der Tastatur wird die Karte verschoben bzw. der Globus gedreht, mit Rechsklick-Ziehen nach oben/unten oder den BildAuf/BildAb-Tasten der Tastatur wird die Ansicht gekippt.
\item Mit einem einfachen Linksklick werden im Detailfenster zusätzliche Daten zum Punkt unter dem Mauszeiger angezeigt. Bei POIs sind das Adresse sowie Beschreibung; bei anderen Punkten genauer Längen- und Breitengrad.
\end{itemize}
\section{Navigationsdaten}
\item Das Zoomlevel kann über einen Schieber angezeigt und geändert werden.
\item Der Längen- und Breitengrad des Punktes an der Bildschirmmitte wird über Eingabefelder angezeigt, und kann über sie geändert werden.
\item Ein Ladebalken zeigt den Fortschritt eventuell im Hintergrund geladener Kartendaten an.

\section{Ansichtseinstellungen}
\begin{itemize}
\item Es kann zwischen 3D- (Globus) und 2D-Ansicht (Aufsicht) gewählt werden.
\item Es besteht eine Auswahl aus verschiedenen Kartentypen; wie Satellitenbildern und Straßenkarten.
\item Wird als Kartentyp die Kinder-Weltkarte gewählt, wir eine Auswahl an angezeigten POIs vorgegeben.
\item Es sind Höhenprofile, -linien, sowie 3D-Ansichten von Häusern und Bäumen zuschaltbar.
\item Es können unter einer Vielzahl an Overlays gewählt werden.
\end{itemize}

\section{Bedienungseinstellungen}
\begin{itemize}
\item Für Linkshänder können die Bedeutungen von linker und rechter Maustaste für die Kartenansicht vertauscht werden.
\end{itemize}

\section{Detailfenster}
\begin{itemize}

\end{itemize}