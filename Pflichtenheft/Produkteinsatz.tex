
\chapter{Produkteinsatz}
\section{Produktvision}
Das Ziel des Projekts besteht darin, eine Desktopanwendung zu entwickeln, die eine 3D-Ansicht der Welt mit Hilfe freier Daten aus dem OpenStreetMap-Projekt bietet. Die grafische Benutzeroberfläche zeigt dafür eine Weltkugel, die frei gedreht und gezoomt werden kann. 

Als Oberflächentextur kann dafür anfangs auf die freien Satellitenbilder der
NASA zurückgegriffen werden, beim hineinzoomen in die Karte wird dann auf eine Kartenansicht von OpenStreetMap gewechselt.

Die einfach zu bedienende Oberfläche bietet eine simple Steuerung via Maus und Tastatur und erlaubt zudem in verschiedene Einstellungen die Darstellung der Welt zu beeinflussen. 


\section{Anwendungsbereich}



\section{Zielgruppe}
Primäre Zielgruppe des Systems sind Privatpersonen (Jugendliche sowie erwachsene Personen), die eine andere Art der Kartendarstellung als die typischen Onlinekarten bevorzugen.

Eine weitere Zielgruppe sollen wissbegierige Kinder darstellen. Voraussetzung ist lediglich der geübte Umgang mit der Maus und/ oder Tastatur. (Eingeschränkte Features)

\section{Betriebsbedingungen}
Lebensdauer, Ausfallsicherheit, Beaufsichtigung(Wartung)

\begin{itemize}
\item Bestehende dauerhafte Internetverbindung zum Laden des Kartenmaterials.
\item Nach  der  Abschlusspräsentation  werden  von  uns  keine  weiteren 
Veränderungen vorgenommen. Es erfolgt keine Wartung durch uns.
\end{itemize} 

\section{Sicherheit, Datenschutz, gesetzliche Vorgaben}
Verwendung von OpenStreetMap, Jogl usw. ist lizenzfrei...